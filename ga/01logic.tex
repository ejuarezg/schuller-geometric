\subsection{Propositional logic}

\bd
A \emph{proposition}\index{proposition} $p$ is a variable\footnote{By this we mean a formal expression, with no extra structure assumed.} that can take the values \emph{true} ($T$) or \emph{false} ($F$), and no others.
\ed

This is what a proposition is from the point of view of propositional logic.
In particular, it is not the task of propositional logic to decide whether a complex statement of the form ``there is extraterrestrial life'' is true or not.
Propositional logic already deals with the complete proposition, and it just assumes that is either true or false.
It is also not the task of propositional logic to decide whether a statement of the type ``in winter is colder than outside'' is a proposition or not (i.e. if it has the property of being either true or false).
In this particular case, the statement looks rather meaningless.

\bd
A proposition which is always true is called a \emph{tautology}\index{tautology}, while one which is always false is called a \emph{contradiction}\index{contradiction}.
\ed

It is possible to build new propositions from given ones using \emph{logical operators}.
The simplest kind of logical operators are \emph{unary} operators, which take in one proposition and return another proposition.
There are four unary operators in total, and they differ by the truth value of the resulting proposition which, in general, depends on the truth value of $p$.
We can represent them in a table as follows:

\btab[h!]
\centering
\btb{c||c|c|c|c}
$p$ & $\neg p$ & $\mathrm{id}(p)$ & $\top p$ & $\bot p$ \\
\hline
\rule{0pt}{12pt} F & T & F & T & F\\
T & F & T & T & F
\etb
\etab
where $\neg$ is the \emph{negation} operator, $\mathrm{id}$ is the \emph{identity} operator, $\top$ is the \emph{tautology} operator and $\bot$ is the \emph{contradiction} operator.
These clearly exhaust all possibilities for unary operators.

The next step is to consider \emph{binary} operators, i.e. operators that take in two propositions and return a new proposition.
There are four combinations of the truth values of two propositions and, since a binary operator assigns one of the two possible truth values to each of those, we have 16 binary operators in total.
The operators $\land$, $\lor$ and~$\veebar$, called \emph{and}, \emph{or} and \emph{exclusive or} respectively, should already be familiar to you.

\btab[h!]
\centering
\btb{c|c||c|c|c}
$p$ & $q$ & $p\land q$ & $p\lor q$ & $p\veebar q$ \\
\hline
\rule{0pt}{12pt} F & F & F & F & F\\
F & T & F & T & T\\
T & F & F & T & T\\
T & T & T & T & F
\etb
\etab

There is one binary operator, the \emph{implication}\index{implication} operator $\imp$, which is sometimes a little ill understood, unless you are already very knowledgeable about these things. Its usefulness comes in conjunction with the \emph{equivalence} operator $\eqv$. We have:

\btab[h!]
\centering
\btb{c|c||c|c}
$p$ & $q$ & $p\imp q$ & $p\eqv q$ \\
\hline
\rule{0pt}{12pt} F & F & T & T \\
F & T & T & F \\
T & F & F & F \\
T & T & T & T 
\etb
\etab

While the fact that the proposition $p\imp q$ is true whenever $p$ is false may be surprising at first, it is just the definition of the implication operator and it is an expression of the principle ``Ex falso quod libet'', that is, from a false assumption anything follows.
Of course, you may be wondering why on earth we would want to define the implication operator in this way.
The answer to this is hidden in the following result.

\bt
Let $p,q$ be propositions.
Then $(p\imp q) \eqv ((\neg q)\imp (\neg p))$.
\et

\bq
We simply construct the truth tables for $p\imp q$ and $ (\neg q)\imp (\neg p)$.

\btab[h!]
\centering
\btb{c|c||c|c|c|c}
$p$ & $q$ & $\neg p$ & $\neg q$ & $p\imp q$ & $(\neg q)\imp (\neg p)$ \\
\hline
\rule{0pt}{12pt} F & F & T & T & T & T\\
F & T & T & F & T & T\\
T & F & F & T & F & F\\
T & T & F & F & T & T
\etb
\etab

The columns for $p\imp q$ and $ (\neg q)\imp (\neg p)$ are identical and hence we are done.
\eq

\br
We agree on decreasing binding strength in the sequence:
\bse
\neg \, , \ \land \, , \ \lor \, , \ \imp \, , \ \eqv .
\ese
For example, $(\neg q)\imp (\neg p)$ may be written unambiguously as $\neg q\imp \neg p$.
\er

\br
All higher order operators $\heartsuit (p_1,\ldots,p_N)$ can be constructed from a single binary operator defined by:

\btab[h!]
\centering
\btb{c|c||c}
$p$ & $q$ & $ p \uparrow q$ \\
\hline
\rule{0pt}{12pt} F & F & T \\
F & T & T \\
T & F & T\\
T & T & F
\etb
\etab

This is called the \emph{nand} operator and, in fact, we have $(p \uparrow q) \eqv \neg (p \land q)$.
\er

\subsection{Predicate logic}

\bd
A \emph{predicate}\index{predicate} is (informally) a proposition-valued function of some variable or variables. In particular, a predicate of two variables is called a \emph{relation}\index{relation}.
\ed

For example, $P(x)$ is a proposition for each choice of the variable $x$, and its truth value depends on $x$.
Similarly, the predicate $Q(x,y)$ is, for any choice of $x$ and $y$, a proposition and its truth value depends on $x$ and $y$.

Just like for propositional logic, it is not the task of predicate logic to examine how predicates are built from the variables on which they depend.
In order to do that, one would need some further language establishing the rules to combine the variables $x$ and $y$ into a predicate.
Also, you may want to specify from which ``set'' $x$ and $y$ come from.
Instead, we leave it completely open, and simply consider $x$ and $y$ formal variables, with no extra conditions imposed.

This may seem a bit weird since from elementary school one is conditioned to always ask where $x$ comes from upon seeing an expression like $P(x)$.
However, it is crucial that we refrain from doing this here, since we want to only later define the notion of set, using the language of propositional and predicate logic.
As with propositions, we can construct new predicates from given ones by using the operators defined in the previous section. For example, we might have:
\bse
Q(x,y,z) :\eqv P(x) \land R(y,z),
\ese
where the symbol $:\eqv$ means ``defined as being equivalent to''.
More interestingly, we can construct a new proposition from a given predicate by using \emph{quantifiers}.
\bd
Let $P(x)$ be a predicate. Then:
\bse
\forall \, x : P(x) ,
\ese
is a proposition, which we read as ``for all $x$, $P$ of $x$ (is true)'', and it is defined to be true if $P(x)$ is true independently of $x$, false otherwise. The symbol $\forall$\index{$\forall$} is called \emph{universal quantifier}\index{universal quantifier}.
\ed
\bd
Let $P(x)$ be a predicate.
Then we define:
\bse
\exists \, x : P(x) : \eqv \neg (\forall \, x : \neg P(x)).
\ese
The proposition $\exists \, x : P(x)$ is read as ``there exists (at least one) $x$ such that $P$ of $x$ (is true)'' and the symbol $\exists$\index{$\exists$} is called \emph{existential quantifier}\index{existential quantifier}.
\ed
The following result is an immediate consequence of these definitions.
\bc
Let $P(x)$ be a predicate. Then:
\bse
\forall \, x : P(x) \eqv \neg (\exists \, x : \neg P(x) ).
\ese
\ec
\br
It is possible to define quantification of predicates of more than one variable.
In order to do so, one proceeds in steps quantifying a predicate of one variable at each step. 
\er
\be
Let $P(x,y)$ be a predicate.
Then, for fixed $y$, $P(x,y)$ is a predicate of one variable and we define:
\bse
Q(y) :\eqv \forall \, x : P(x,y).
\ese
Hence we may have the following:
\bse
\exists \, y : \forall \, x : P(x,y) :\eqv \exists \, y : Q(y).
\ese
Other combinations of quantifiers are defined analogously.
\ee
\br
The order of quantification matters (if the quantifiers are not all the same).
For a given predicate $P(x,y)$, the propositions:
\bse
\exists \, y : \forall \, x : P(x,y)  \quad \text{and} \quad \forall \, x : \exists \, y : P(x,y) 
\ese
are not necessarily equivalent.
\er
\be
Consider the proposition expressing the existence of additive inverses in the real numbers. We have:
\bse
\forall \, x : \exists \, y : \ x+y=0,
\ese
i.e. for each $x$ there exists an inverse $y$ such that $x+y=0$. For $1$ this is $-1$, for $2$ it is $-2$ etc.
Consider now the proposition obtained by swapping the quantifiers in the previous proposition:
\bse
\exists \, y : \forall \, x : \ x+y=0.
\ese

What this proposition is saying is that there exists a real number $y$ such that, no matter what $x$ is, we have $x+y=0$.
This is clearly false, since if $x+y=0$ for some $x$ then $(x+1)+y\neq 0$, so the same $y$ cannot work for both $x$ and $x+1$, let alone every $x$.
\ee
Notice that the proposition $\exists \, x : P(x)$ means ``there exists \emph{at least one} $x$ such that $P(x)$ is true''. Often in mathematics we prove that ``there exists \emph{a unique} $x$ such that $P(x)$ is true''. We therefore have the following definition.

\bd
Let $P(x)$ be a predicate. We define the \emph{unique existential quantifier} $\exists !$ by:
\bse
\exists ! \, x : P(x) :\eqv (\exists \, x : P(x)) \land \forall \, y : \forall \, z : (P(y)\land P(z) \imp y=z) .
\ese
\ed
This definition clearly separates the existence condition from the uniqueness condition. An equivalent definition with the advantage of brevity is:
\bse
\exists ! \, x : P(x) :\eqv \exists \, x : \forall \, y : (P(y) \eqv x=y)
\ese

\subsection{Axiomatic systems and theory of proofs}

\bd
An \emph{axiomatic system} is a finite sequence of propositions $a_1,a_2,\ldots,a_N$, which are called the \emph{axioms}\index{axiom} of the system.
\ed

\bd
A \emph{proof}\index{proof} of a proposition $p$ within an axiomatic system $a_1,a_2,\ldots,a_N$ is a finite sequence of propositions $q_1,q_2,\ldots,q_M$ such that $q_M=p$ and for any $1\leq j \leq M$ one of the following is satisfied:
\ben
\item[(A)] $q_j$ is a proposition from the list of axioms;
\item[(T)] $q_j$ is a tautology;
\item[(M)] $\exists \, 1\leq m,n <j : (q_m\land q_n \imp q_j)$ is true.
\een
\ed
\br
If $p$ can be proven within an axiomatic system $a_1,a_2,\ldots,a_N$, we write:
\bse
a_1,a_2,\ldots,a_N \vdash p
\ese
and we read ``$a_1,a_2,\ldots,a_N$ proves $p$''.
\er
\br
This definition of proof allows to easily recognise a proof.
A computer could easily check that whether or not the conditions (A), (T) and (M) are satisfied by a sequence of propositions.
To actually find a proof of a proposition is a whole different story.
\er
\br
Obviously, any tautology that appears in the list of axioms of an axiomatic system can be removed from the list without impairing the power of the axiomatic system.  
\er

An extreme case of an axiomatic system is propositional logic.
The axiomatic system for propositional logic is the empty sequence.
This means that all we can prove in propositional logic are tautologies. This
proof is given for \Cref{thm:prop_logic_is_consitent}.

\bd
An axiomatic system $a_1,a_2,\ldots,a_N$ is said to be \emph{consistent}\index{consistency} if there exists a proposition $q$ which cannot be proven from the axioms.
In symbols:
\bse
\exists \, q : \neg (a_1,a_2,\ldots,a_N \vdash q).
\ese
\ed

The idea behind this definition is the following.
Consider an axiomatic system which contains contradicting propositions:
\bse
a_1,\ldots,s,\ldots,\neg s,\ldots,a_N .
\ese
Then, given \emph{any} proposition $q$, the following is a proof of $q$ within this system:
\bse
s, \neg s, q.
\ese
Indeed, $s$ and $\neg s$ are legitimate steps in the proof since they are axioms.
Moreover, $s\land \neg s$ is a contradiction and thus $(s\land \neg s) \imp q$ is a tautology.
Therefore, $q$ follows from condition (M).
This shows that any proposition can be proven within a system with contradictory axioms.
In other words, the inability to prove every proposition is a property possessed by no contradictory system, and hence we define a consistent system as one with this property.

Having come this far, we can now state (and prove) an impressively sounding theorem.

\bt
Propositional logic is consistent. \label{thm:prop_logic_is_consitent}
\et

\bq
Suffices to show that there exists a proposition that cannot be proven within propositional logic. Propositional logic has the empty sequence as axioms.
Therefore, only conditions (T) and (M) are relevant here.
The latter allows the insertion of a proposition $q_j$ such that $(q_m\land q_n) \imp q_j$ is true, where $q_m$ and $q_n$ are propositions that precede $q_j$ in the proof sequence.
However, since (T) only allows the insertion of a tautology anywhere in the proof sequence, the propositions $q_m$ and $q_n$ must be tautologies.
Consequently, for $(q_m\land q_n) \imp q_j$ to be true, $q_j$ must also be a tautology.
Hence, the proof sequence consists entirely of tautologies and thus only tautologies can be proven.

Now let $q$ be any proposition.
Then $q\land \neg q$ is a contradiction, hence not a tautology and thus cannot be proven.
Therefore, propositional logic is consistent.
\eq

\br
While it is perfectly fine and clear how to define consistency, it is perfectly difficult to prove consistency for a given axiomatic system, propositional logic being a big exception.
\er

\bt
Any axiomatic system powerful enough to encode elementary arithmetic is either inconsistent or contains an \emph{undecidable}\index{undecidable} proposition, i.e.\ a proposition that can be neither proven nor disproven within the system.
\et

An example of an undecidable proposition is the Continuum hypothesis within the Zermelo-Fraenkel axiomatic system.  
