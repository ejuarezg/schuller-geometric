\subsection{Differential forms}


\bd
Let $M$ be a smooth manifold. A \emph{(differential) $n$-form}\index{differential form} on $M$ is a $(0,n)$ smooth tensor field $\omega$ which is totally antisymmetric, i.e.\
\bse
\omega(X_1,\ldots,X_n) = \sgn(\pi)\, \omega(X_{\pi(1)},\ldots,X_{\pi(n)}),
\ese
for any $\pi \in S_n$, with $X_i\in \Gamma(TM)$.  
\ed

Alternatively, we can define a differential form as a smooth section of the appropriate bundle on $M$, i.e.\ as a map assigning to each $p\in M$ an $n$-form on the vector space $T_pM$.

\be
\ben[label=\alph*)]
\item A manifold $M$ is said to be \emph{orientable} if it admits an oriented atlas, i.e.\ an atlas in which all chart transition maps, which are maps between open subsets of $\R^{\dim M}$, have a positive determinant.

If $M$ is orientable, then there exists a nowhere vanishing top form ($n=\dim M$) on $M$ providing the volume.
\item The electromagnetic field strength $F$ is a differential $2$-form built from the electric and magnetic fields, which are also taken to be forms. We will define these later in some detail.
\item In classical mechanics, if $Q$ is a smooth manifold describing the possible system configurations, then the phase space is $T^*Q$. There exists a canonically defined $2$-form on $T^*Q$ known as a symplectic form, which we will define later.
\een
\ee

If $\omega$ is an $n$-form, then $n$ is said to be the \emph{degree} of $\omega$. We denote by $\Omega^n(M)$ the set of all differential $n$-forms on $M$, which then becomes a $\mathcal{C}^\infty(M)$-module by defining the addition and multiplication operations pointwise.

\be
Of course, we have $\Omega^0(M)\equiv \mathcal{C}^\infty(M)$ and $\Omega^1(M)\equiv \Gamma(T^0_1M)\equiv\Gamma(T^*M)$.
\ee

Similarly to the case of forms on vector spaces, we have $\Omega^n(M)=\{0\}$ for $n>\dim M$, and otherwise $\dim \Omega^n(M)= {{\dim M}\choose{n}}$, as a $\mathcal{C}^\infty(M)$-module.

We can specialise the pull-back of tensors to differential forms.

\bd
Let $\phi\cl M \to N$ be a smooth map and let $\omega\in \Omega^n(N)$. Then we define the \emph{pull-back} $\Phi^*(\omega)\in \Omega^n(M)$ of $\omega$ as
\bi{rrCl}
\Phi^*(\omega)\cl & M &\to & T^*M\\
& p & \mapsto & \Phi^*(\omega) (p), 
\ei
where
\bse
\Phi^*(\omega)(p)(X_1,\ldots,X_n):=\omega (\phi(p))\bigl( \phi_*(X_1),\ldots,\phi_*(X_n) \bigr),
\ese
for $X_i\in T_pM$.
\ed
The map $\Phi^*\cl\Omega^n(N)\to\Omega^n(M)$ is $\R$-linear, and its action on $\Omega^0(M)$ is simply
\bi{rrCl}
\Phi^*\cl & \Omega^0(M) & \to & \Omega^0(M)\\
&f &\mapsto &\Phi^*(f) := f\circ \phi.
\ei
This works for any smooth map $\phi$, and it leads to a slight modification of our mantra:
\begin{center}
\textit{Vectors are pushed forward,}\\
\textit{forms are pulled back.}
\end{center}
The tensor product $\otimes$ does not interact well with forms, since the tensor product of two forms is not necessarily a form. Hence, we define the following.

\bd
Let $M$ be a smooth manifold. We define the \emph{wedge}\index{wedge product} (or \emph{exterior}) \emph{product} of forms as the map
\bi{rrCl}
\wedge \cl & \Omega^n(M) \times \Omega^m(M) & \to & \Omega^{n+m}(M)\\
& (\omega,\sigma) & \mapsto & \omega \wedge \sigma,
\ei
where
\bse
(\omega\wedge\sigma)(X_1,\ldots,X_{n+m}) := \frac{1}{n!\,m!} \sum_{\pi \in S_{n+m}} \sgn(\pi) (\omega \otimes \sigma)(X_{\pi(1)},\ldots,X_{\pi(n+m)})
\ese
and $X_1,\ldots,X_{n+m}\in\Gamma(TM)$. By convention, for any $f,g\in \Omega^0(M)$ and $\omega\in \Omega^n(M)$, we set
\bse
f\wedge g := fg \qquad \text{and} \qquad f\wedge\omega=\omega\wedge f = f\omega.
\ese
\ed
\be
Suppose that $\omega,\sigma\in\Omega^1(M)$. Then, for any $X,Y\in\Gamma(TM)$
\bi{rCl}
(\omega\wedge\sigma)(X,Y) & = & (\omega\otimes\sigma)(X,Y) - (\omega\otimes\sigma)(Y,X)\\
& = & (\omega\otimes\sigma)(X,Y) - \omega(Y)\sigma(X)\\
& = & (\omega\otimes\sigma)(X,Y) - (\sigma \otimes \omega)(X,Y)\\
& = & (\omega\otimes\sigma -\sigma \otimes \omega)(X,Y).
\ei
Hence
\bse
\omega\wedge\sigma = \omega\otimes\sigma - \sigma \otimes \omega. 
\ese
\ee
The wedge product is bilinear over $\mathcal{C}^\infty(M)$, that is
\bse
(f\omega_1 +\omega_2)\wedge \sigma = f\, \omega_1\wedge\sigma+\omega_2\wedge\sigma,
\ese
for all $f\in\mathcal{C}^\infty(M)$, $\omega_1,\omega_2\in\Omega^n(M)$ and $\sigma\in\Omega^m(M)$, and similarly for the second argument.

\br
If $(U,x)$ is a chart on $M$, then every $n$-form $\omega\in \Omega^n(U)$ can be expressed locally on $U$ as
\bse
\omega = \omega_{a_1\cdots a_n} \, \d x^{a_1} \wedge \cdots \wedge \d x^{a_n},
\ese
where $\omega_{a_1\cdots a_n}\in \mathcal{C}^\infty(U)$ and $1\leq a_1 < \cdots < a_n \leq \dim M$. The $\d x^{a_i}$ appearing above are the covector fields ($1$-forms)
\bse
\d x^{a_i} \cl p\mapsto \d_px^{a_i}.
\ese
\er

The pull-back distributes over the wedge product.

\begin{theorem}
Let $\phi\cl M\to N$ be smooth, $\omega\in\Omega^n(N)$ and $\sigma\in\Omega^m(N)$. Then, we have
\bse
\Phi^*(\omega\wedge\sigma) = \Phi^*(\omega)\wedge\Phi^*(\sigma).  
\ese
\end{theorem}

\bq
Let $p\in M$ and $X_1,\ldots,X_{n+m}\in T_pM$. Then we have
\bi{rCl}
\IEEEeqnarraymulticol{3}{l}{\bigl( \Phi^*(\omega)\wedge\Phi^*(\sigma) \bigr)  (p)(X_1,\ldots,X_{n+m}) }\\
\qquad \qquad & = & \frac{1}{n!\,m!} \sum_{\pi \in S_{n+m}} \sgn(\pi) \bigl( \Phi^*(\omega)\otimes\Phi^*(\sigma) \bigr)(p)(X_{\pi(1)},\ldots,X_{\pi(n+m)})\\
& = & \frac{1}{n!\,m!} \sum_{\pi \in S_{n+m}} \sgn(\pi)\Phi^*(\omega)(p)(X_{\pi(1)},\ldots,X_{\pi(n)})\\[-10pt]
& & \hspace{3.9cm} \Phi^*(\sigma) (p)(X_{\pi(n+1)},\ldots,X_{\pi(n+m)})\\
& = & \frac{1}{n!\,m!} \sum_{\pi \in S_{n+m}} \sgn(\pi)\omega(\phi(p))\bigl( \phi_*(X_{\pi(1)}),\ldots,\phi_*(X_{\pi(n)})\bigr) \\[-10pt]
& & \hspace{4.4cm} \sigma(\phi(p))\bigl( \phi_*(X_{\pi(n+1)}),\ldots,\phi_*(X_{\pi(n+m)})\bigr)\\
& = & \frac{1}{n!\,m!} \sum_{\pi \in S_{n+m}}\sgn(\pi) (\omega\otimes\sigma) (\phi(p)) \bigl( \phi_*(X_{\pi(1)}),\ldots,\phi_*(X_{\pi(n+m)})\bigr)\\
& = & (\omega\wedge\sigma) (\phi(p)) \bigl( \phi_*(X_1),\ldots,\phi_*(X_{n+m})\bigr)\\
& = & \Phi^*(\omega\wedge\sigma) (p) ( X_1,\ldots,X_{n+m}).
\ei
% \bi{rCl}
% \IEEEeqnarraymulticol{3}{l}{\bigl( \Phi^*(\omega)\wedge\Phi^*(\sigma) \bigr)  (p)(X_1,\ldots,X_{n+m}) }\\
% \qquad & = & \frac{1}{n!\,m!} \sum_{\pi \in S_{n+m}} \sgn(\pi) \bigl( \Phi^*(\omega)\otimes\Phi^*(\sigma) \bigr)(p)(X_{\pi(1)},\ldots,X_{\pi(n+m)})\\
% & = & \frac{1}{n!\,m!} \sum_{\pi \in S_{n+m}} \sgn(\pi)\Phi^*(\omega)(p)(X_{\pi(1)},\ldots,X_{\pi(n)})\,\Phi^*(\sigma) (p)(X_{\pi(n+1)},\ldots,X_{\pi(n+m)})\\
% & = & \frac{1}{n!\,m!} \sum_{\pi \in S_{n+m}} \sgn(\pi)\omega(\phi(p))\bigl( \phi_*(X_{\pi(1)}),\ldots,\phi_*(X_{\pi(n)})\bigr) \,\sigma(\phi(p))\bigl( \phi_*(X_{\pi(n+1)}),\ldots,\phi_*(X_{\pi(n+m)})\bigr)\\
% & = & \frac{1}{n!\,m!} \sum_{\pi \in S_{n+m}}\sgn(\pi) (\omega\otimes\sigma) (\phi(p)) \bigl( \phi_*(X_{\pi(1)}),\ldots,\phi_*(X_{\pi(n+m)})\bigr)\\
% & = & (\omega\wedge\sigma) (\phi(p)) \bigl( \phi_*(X_1),\ldots,\phi_*(X_{n+m})\bigr)\\
% & = & \Phi^*(\omega\wedge\sigma) (p) ( X_1,\ldots,X_{n+m}).
% \ei
Since $p\in M$ was arbitrary, the statement follows.
\eq

\subsection{The Grassmann algebra}

Note that the wedge product takes two differential forms and produces a differential form of a different type. It would be much nicer to have a space which is closed under the action of $\wedge$. In fact, such a space exists and it is called the Grassmann algebra of $M$.

\bd
Let $M$ be a smooth manifold. Define the $\mathcal{C}^\infty(M)$-module
\bse
\Gr(M)\equiv\Omega(M) := \bigoplus_{n=0}^{\dim M} \Omega^n(M).
\ese
The \emph{Grassmann algebra}\index{Grassmann algebra} on $M$ is the algebra $(\Omega(M),+,\cdot,\wedge)$, where
\bse
\wedge\cl \Omega(M)\times\Omega(M)\to\Omega(M)
\ese
is the linear continuation of the previously defined $\wedge\cl\Omega^n(M)\times\Omega^m(M)\to\Omega^{n+m}(M)$.
\ed
Recall that the direct sum of modules has the Cartesian product of the modules as underlying set and module operations defined componentwise. Also, note that by ``algebra'' here we really mean ``algebra over a module''.

\be
Let $\psi=\omega+\sigma$, where $\omega\in\Omega^1(M)$ and $\sigma\in\Omega^3(M)$. Of course, this ``+'' is neither the addition on $\Omega^1(M)$ nor the one on $\Omega^3(M)$, but rather that on $\Omega(M)$ and, in fact, $\psi\in\Omega(M)$. 

Let $\varphi\in \Omega^n(M)$, for some $n$. Then
\bse
\varphi\wedge\psi=\varphi\wedge(\omega+\sigma)= \varphi\wedge\omega+\varphi\wedge\sigma,
\ese
where $\varphi\wedge\omega\in\Omega^{n+1}(M)$, $\varphi\wedge\sigma\in\Omega^{n+3}(M)$, and $\varphi\wedge\psi\in\Omega(M)$.
\ee

\be
There is a lot of talk about \emph{Grassmann numbers}, particularly in supersymmetry. One often hears that these are ``numbers that do not commute, but anticommute''. Of course, objects cannot be commutative or anticommutative by themselves. These qualifiers only apply to operations on the objects.
In fact, the Grassmann numbers are just the elements of a Grassmann algebra. 
\ee

The following result is about the anticommutative behaviour of $\wedge$. 

\begin{theorem}
Let $\omega\in\Omega^n(M)$ and $\sigma\in\Omega^m(M)$. Then
\bse
\omega\wedge\sigma=(-1)^{nm}\,\sigma\wedge\omega.
\ese
\end{theorem}

We say that $\wedge$ is \emph{graded commutative}, that is, it satisfies a version of anticommutativity which depends on the degrees of the forms.

\bq
First note that if $\omega,\sigma\in\Omega^1(M)$, then
\bse
\omega\wedge\sigma = \omega\otimes\sigma - \sigma \otimes \omega = \negmedspace {}- \sigma \wedge \omega. 
\ese
Recall that is $\omega\in\Omega^n(M)$ and $\sigma\in\Omega^m(M)$, then locally on a chart $(U,x)$ we can write
\bi{C}
\omega = \omega_{a_1\cdots a_n}\d x^{a_1}\wedge \cdots \wedge \d x^{a_n} \\
\sigma = \sigma_{b_1\cdots b_m}\d x^{b_1}\wedge \cdots \wedge \d x^{b_m}
\ei
with $1\leq a_1 < \cdots < a_n \leq \dim M$ and similarly for the $b_i$. The coefficients $\omega_{a_1\cdots a_n}$ and $\sigma_{b_1\cdots b_m}$ are smooth functions in $\mathcal{C}^\infty(U)$. Since $\d x^{a_i},\d x^{b_j}\in \Omega^1(M)$, we have
\bi{rCl}
\omega\wedge\sigma & = & \omega_{a_1\cdots a_n}\sigma_{b_1\cdots b_m}\,\d x^{a_1}\wedge \cdots \wedge \d x^{a_n}\wedge\d x^{b_1}\wedge \cdots \wedge \d x^{b_m}\\
& = &(-1)^n\,\omega_{a_1\cdots a_n}\sigma_{b_1\cdots b_m}\,\d x^{b_1}\wedge\d x^{a_1}\wedge \cdots \wedge \d x^{a_n}\wedge\d x^{b_2}\wedge \cdots \wedge \d x^{b_m}\\
& = &(-1)^{2n}\,\omega_{a_1\cdots a_n}\sigma_{b_1\cdots b_m}\,\d x^{b_1}\wedge\d x^{b_2}\wedge\d x^{a_1}\wedge \cdots \wedge \d x^{a_n}\wedge\d x^{b_3}\wedge \cdots \wedge \d x^{b_m}\\
& \vdots &\\
& = &(-1)^{nm}\,\omega_{a_1\cdots a_n}\sigma_{b_1\cdots b_m}\,\d x^{b_1}\wedge \cdots \wedge \d x^{b_m}\wedge\d x^{a_1}\wedge \cdots \wedge \d x^{a_n}\\
& = & (-1)^{nm} \, \sigma \wedge \omega
\ei
since we have swapped $1$-forms $nm$-many times.
\eq

\br
We should stress that this is only true when $\omega$ and $\sigma$ are pure degree forms, rather than linear combinations of forms of different degrees. Indeed, if $\varphi,\psi\in\Omega(M)$, a formula like
\bse
\varphi\wedge\psi = \cdots \psi \wedge \varphi
\ese
does not make sense in principle, because the different parts of $\varphi$ and $\psi$ can have different commutation behaviours.
\er

\subsection{The exterior derivative}

Recall the definition of the gradient operator at a point $p\in M$. We can extend that definition to define the ($\R$-linear) operator:
\bi{rrCl}
\d \cl & \mathcal{C}^\infty(M) & \xrightarrow{\sim} & \Gamma(T^*M)\\
& f & \mapsto & \d f
\ei
where, of course, $\d f \cl p \mapsto \d_pf$. Alternatively, we can think of $\d f$ as the $\R$-linear map
\bi{rrCl}
\d f \cl & \Gamma(TM) & \xrightarrow{\sim} & \mathcal{C}^\infty(M)\\
& X &\mapsto & \d f (X) = X(f).
\ei

\br
Locally on some chart $(U,x)$ on $M$, the covector field (or $1$-form) $\d f$ can be expressed as
\bse
\d f = \lambda_a\, \d x^a
\ese
for some smooth functions $\lambda_i\in\mathcal{C}^\infty(U)$. To determine what they are, we simply apply both sides to the vector fields induced by the chart. We have
\bse
\d f \biggl( \frac{\partial}{\partial x^b}\biggr) =  \frac{\partial}{\partial x^b}(f) = \partial_bf
\ese
and
\bse
\lambda_a\, \d x^a\biggl( \frac{\partial}{\partial x^b}\biggr) =\lambda_a\,  \frac{\partial}{\partial x^b}(x^a) = \lambda_a\,\delta^a_b = \lambda_b.
\ese
Hence, the local expression of $\d f$ on $(U,x)$ is
\bse
\d f = \partial_a f \, \d x^a.
\ese
\er
Note that the operator $\d$ satisfies the Leibniz rule
\bse
\d (fg) = g\, \d f + f\, \d g.
\ese
We can also understand this as an operator that takes in $0$-forms and outputs $1$-forms
\bse
\d \cl \Omega^0(M)\xrightarrow{\sim}\Omega^1(M).
\ese
This can then be extended to an operator which acts on any $n$-form. We will need the following definition.
\bd
Let $M$ be a smooth manifold and let $X,Y\in\Gamma(TM)$. The commutator (or Lie bracket\index{Lie bracket}) of $X$ and $Y$ is defined as
\bi{rrCl}
[X,Y]\cl & \mathcal{C}^\infty(M) &\xrightarrow{\sim} &\mathcal{C}^\infty(M)\\ 
& f & \mapsto & [X,Y](f):=X(Y(f))-Y(X(f)),
\ei
where we are using the definition of vector fields as $\R$-linear maps $\mathcal{C}^\infty(M) \xrightarrow{\sim} \mathcal{C}^\infty(M)$.
\ed

\bd
The \emph{exterior derivative}\index{exterior derivative} on $M$ is the $\R$-linear operator
\bi{rrCl}
\d\cl &\Omega^n(M)&\xrightarrow{\sim} &\Omega^{n+1}(M)\\
& \omega & \mapsto & \d \omega
\ei
with $\d \omega$ being defined as
\bi{rCl}
\d \omega (X_1,\ldots,X_{n+1}) & := & \sum_{i=1}^{n+1} (-1)^{i+1}\, X_i \bigl(\omega(X_1,\ldots,\widehat{X_i},\ldots,X_{n+1})\bigr)\\
& & {} \negmedspace + \sum_{i<j} (-1)^{i+j}\,\omega\bigl([X_i,X_j],X_1,\ldots,\widehat{X_i},\ldots,\widehat{X_j},\ldots,X_{n+1}\bigr),
\ei
where $X_i\in \Gamma(TM)$ and the hat denotes omissions.
\ed

\br
Note that the operator $\d$ is only well-defined when it acts on forms. In order to define a derivative operator on general tensors we will need to add extra structure to our differentiable manifold.
\er

\be
In the case $n=1$, the form $\d \omega\in\Omega^2(M)$ is given by
\bse
\d \omega (X,Y) = X(\omega(Y))-Y(\omega(X))-\omega([X,Y]).
\ese
Let us check that this is indeed a $2$-form, i.e.\ an antisymmetric, $\mathcal{C}^\infty(M)$-multilinear map
\bse
\d\omega\cl \Gamma(TM)\times\Gamma(TM)\to\mathcal{C}^\infty(M).
\ese
By using the antisymmetry of the Lie bracket, we immediately get
\bse
\d \omega(X,Y)= - \,\d\omega(Y,X).
\ese
Moreover, thanks to this identity, it suffices to check $\mathcal{C}^\infty(M)$-linearity in the first argument only. Additivity is easily checked
\bi{rCl}
\d\omega(X_1+X_2,Y) & = & (X_1+X_2)(\omega(Y))-Y(\omega(X_1+X_2))-\omega([X_1+X_2,Y])\\
& = & X_1(\omega(Y))+X_2(\omega(Y))-Y(\omega(X_1)+\omega(X_2))-\omega([X_1,Y]+[X_2,Y])\\
& = & X_1(\omega(Y))+X_2(\omega(Y))-Y(\omega(X_1))-Y(\omega(X_2))-\omega([X_1,Y])-\omega([X_2,Y])\\
&=& \d\omega(X_1,Y) +\d\omega(X_2,Y).
\ei
For $\mathcal{C}^\infty(M)$-scaling, first we calculate $[fX,Y]$. Let $g\in\mathcal{C}^\infty(M)$. Then
\bi{rCl}
[fX,Y](g) &=& fX(Y(g)) -Y (fX(g))\\
&=& fX(Y(g)) - fY(X(g)) -Y (f)X(g)\\
&=& f(X(Y(g)) - Y(X(g))) -Y (f)X(g)\\
&=& f[X,Y](g) -Y(f)X(g)\\
&=& (f[X,Y] -Y(f)X)(g).
\ei
Therefore
\bse
[fX,Y]=f[X,Y]-Y(f)X.
\ese
Hence, we can calculate
\bi{rCl}
\d\omega(fX,Y) & = & fX(\omega(Y))-Y(\omega(fX))-\omega([fX,Y])\\
& = & fX(\omega(Y))-Y(f\omega(X))-\omega(f[X,Y]-Y(f)X)\\
& = & fX(\omega(Y))-fY(\omega(X))-Y(f)\omega(X)-f\omega([X,Y])+\omega(Y(f)X)\\
& = & fX(\omega(Y))-fY(\omega(X))-\Ccancel[gray]{Y(f)\omega(X)}-f\omega([X,Y])+\Ccancel[gray]{Y(f)\omega(X)}\\
&=& f\,\d\omega(X,Y), 
\ei
which is what we wanted.
\ee

The exterior derivative satisfies a graded version of the Leibniz rule with respect to the wedge product.

\begin{theorem}
Let $\omega\in\Omega^n(M)$ and $\sigma\in\Omega^m(M)$. Then
\bse
\d(\omega\wedge\sigma)=\d\omega\wedge\sigma+(-1)^n\,\omega\wedge\d\sigma.
\ese
\end{theorem}

\bq
We will work in local coordinates. Let $(U,x)$ be a chart on $M$ and write
\bi{C}
\omega = \omega_{a_1\cdots a_n}\d x^{a_1}\wedge \cdots \wedge \d x^{a_n} =: \omega_A \d x^A\\
\sigma = \sigma_{b_1\cdots b_m}\d x^{b_1}\wedge \cdots \wedge \d x^{b_m} =: \sigma_B \d x^B.
\ei
Locally, the exterior derivative operator $\d$ acts as
\bse
\d \omega = \d \omega_A \wedge \d x^A.
\ese
Hence
\bi{rCl}
\d (\omega\wedge\sigma) & = & \d (\omega_A\sigma_B\, \d x^A\wedge \d x^B)\\
 & = & \d (\omega_A\sigma_B)\wedge\d x^A\wedge \d x^B\\
 & = & (\sigma_B \d \omega_A +\omega_A\d\sigma_B)\wedge\d x^A\wedge \d x^B\\
 & = & \sigma_B \d \omega_A\wedge\d x^A\wedge \d x^B+\omega_A\d\sigma_B \wedge\d x^A\wedge \d x^B\\
 & = & \sigma_B \d \omega_A\wedge\d x^A\wedge \d x^B+(-1)^n\omega_A\d x^A\wedge\d\sigma_B \wedge \d x^B\\
 & = & \sigma_B \d \omega\wedge \d x^B+(-1)^n\omega_A\d x^A\wedge\d\sigma\\
 & = & \d \omega\wedge  \sigma+(-1)^n\, \omega\wedge\d\sigma
\ei
since we have ``anticommuted'' the $1$-form $\d\sigma_B$ through the $n$-form $\d x^A$, picking up $n$ minus signs in the process.
\eq

An important property of the exterior derivative is the following.

\begin{theorem}
Let $\phi\cl M\to N$ be smooth. For any $\omega\in\Omega^n(N)$, we have
\bse
\Phi^*(\d \omega) = \d (\Phi^*(\omega)).
\ese
\end{theorem}


\bq[Proof (sketch)] 
We first show that this holds for $0$-forms (i.e.\ smooth functions).

Let $f\in\mathcal{C}^\infty(N)$, $p\in M$ and $X\in T_pM$. Then
\bi{rCl"s}
\Phi^*(\d f)(p)(X) & = & \d f (\phi(p)) (\phi_*(X)) & (definition of $\Phi^*$)\\
& = & \phi_*(X) (f) & (definition of $\d f$)\\
& = & X (f\circ \phi) & (definition of $\phi_*$)\\
& = & \d (f\circ \phi) (p) (X) & (definition of $\d (f\circ\phi$))\\
& = & \d (\Phi^*(f)) (p) (X) & (definition of $\Phi^*$),
\ei
so that we have $\Phi^*(\d f) = \d (\Phi^*(f))$.

The general result follows from the linearity of $\Phi^*$ and the fact that the pull-back distributes over the wedge product.
\eq

\br
Informally, we can write this result as $\Phi^*\d=\d\Phi^*$, and say that the exterior derivative ``commutes'' with the pull-back.

However, you should bear in mind that the two $\d$'s appearing in the statement are two different operators. On the left hand side, it is $\d\cl\Omega^n(N)\to\Omega^{n+1}(N)$, while it is $\d\cl\Omega^n(M)\to\Omega^{n+1}(M)$ on the right hand side.
\er

\br
Of course, we could also combine the operators $\d$ into a single operator acting on the Grassmann algebra on $M$
\bse
\d \cl \Omega(M)\to\Omega(M)
\ese
by linear continuation.
\er

\be
In the modern formulation of Maxwell's electrodynamics, the electric and magnetic fields $E$ and $B$ are taken to be a $1$-form and a $2$-form on $\R^3$, respectively:
\bi{c}
E := E_x\d x + E_y \d y +E_z \d z\\
B := B_x\d y \wedge \d z + B_y \d z \wedge \d x + B_z \d x \wedge \d y.  
\ei
The electromagnetic field strength $F$ is then defined as the $2$-form on $\R^4$
\bse
F:= B + E \wedge \d t.
\ese
In components, we can write
\bse
F = F_{\mu\nu} \, \d x^\mu \wedge \d x^\nu,
\ese
where $(\d x^0,\d x^1, \d x^2, \d x^3) \equiv (\d t, \d x,\d y,\d z)$ and
\bse
F_{\mu\nu} = 
\left(
\begin{matrix}
0 & -E_x & -E_y & -E_z\\
E_x & 0 & B_z & -B_y\\
E_y & -B_z & 0 & B_x\\
E_z & B_y & -B_x & 0
\end{matrix}
\right)
\ese

The field strength satisfies the equation
\bse
\d F = 0.
\ese
This is called the homogeneous Maxwell's equation and it is, in fact, equivalent to the two homogeneous Maxwell's (vectorial) equations
\bse
\nabla \cdot \mathbf{B} = 0
\ese
\bse
\nabla \times \mathbf{E} + \frac{\partial \mathbf{B}}{\partial t} = \mathbf{0}.
\ese
In order to cast the remaining Maxwell's equations into the language of differential forms, we need a further operation on forms, called the Hodge star operator.

Recall from the standard theory of electrodynamics that the two equations above imply the existence of the electric and vector potentials $\varphi$ and $\mathbf{A}=(A_x,A_y,A_z)$, satisfying
\bse
\mathbf{B} = \nabla \times \mathbf{A} 
\ese
\bse
\mathbf{E} = -\nabla \varphi - \frac{\partial \mathbf{A}}{\partial t}.
\ese
Similarly, the equation $\d F =0$ on $\R^4$ implies the existence of an electromagnetic $4$-potential (or gauge potential) form $A\in\Omega^1(\R^4)$ such that
\bse
F = \d A.
\ese
Indeed, we can take
\bse
A := -\varphi \, \d t + A_x \d x + A_y \d y + A_z \d z.
\ese
\ee

\bd
Let $M$ be a smooth manifold. A $2$-form $\omega\in\Omega^2(M)$ is said to be a \emph{symplectic form}\index{symplectic form} on $M$ if $\d \omega = 0$ and if it is non-degenerate, i.e.\
\bse
(\forall \, Y\in \Gamma(TM) : \omega(X,Y) = 0) \Rightarrow X = 0.
\ese
A manifold equipped with a symplectic form is called a \emph{symplectic manifold}.
\ed

\be
In the Hamiltonian formulation of classical mechanics one is especially interested in the cotangent bundle $T^*Q$ of some configuration space $Q$. Similarly to what we did when we introduced the tangent bundle, we can define (at least locally) a system of coordinates on $T^*Q$ by
\bse
(q^1,\ldots,q^{\dim Q},p_1,\ldots,p_{\dim Q}),
\ese
where the $p_i$'s are the generalised momenta on $Q$ and the $q^i$'s are the generalised coordinates on $Q$ (recall that $\dim T^*Q=2\dim Q$). We can then define a $1$-form $\theta\in\Omega^1(T^*Q)$ by
\bse
\theta := p_i \, \d q^i
\ese
called the symplectic potential. If we further define
\bse
\omega := \d \theta \in \Omega^2(T^*Q),
\ese
then we can calculate that
\bse
\d \omega = \d (\d \theta) =\cdots = 0.
\ese
Moreover, $\omega$ is non-degenerate and hence it is a symplectic form on $T^*Q$.
\ee

\subsection{de Rham cohomology}


The last two examples suggest two possible implications. In the electrodynamics example, we saw that
\bse
(\d F = 0 )\Rightarrow (\exists \, A : F =\d A),
\ese
while in the Hamiltonian mechanics example we saw that
\bse
(\exists\, \theta : \omega =\d \theta )\Rightarrow (\d \omega = 0).
\ese
\bd
Let $M$ be a smooth manifold and let $\omega \in \Omega^n(M)$. We say that $\omega$ is
\begin{itemize}
\item \emph{closed}\index{closed form} if $\d \omega = 0$;
\item \emph{exact}\index{exact form} if $\exists \, \sigma \in \Omega^{n-1}(M): \omega = \d \sigma$.
\end{itemize}
\ed
The question of whether every closed form is exact and vice versa, i.e.\ whether the implications 
\bse
(\d \omega =0) \Leftrightarrow (\exists \, \sigma : \omega =\d \sigma)
\ese
hold in general, belongs to the branch of mathematics called cohomology theory, to which we will now provide an introduction.

The answer for the $\Leftarrow$ direction is affirmative thanks to the following result.

\begin{theorem}
Let $M$ be a smooth manifold. The operator
\bse
\d^2 \equiv \d \circ \d \cl \Omega^n(M) \to \Omega^{n+2}(M)
\ese
is identically zero, i.e.\ $\d^2=0$.
\end{theorem}

For the proof, we will need the following concepts.

\bd
Given an object which carries some indices, say $T_{a_1,\ldots,a_n}$, we define the \emph{antisymmetrization} of $T_{a_1,\ldots,a_n}$ as
\bse
T_{[a_1\cdots a_n]}:=\frac{1}{n!}\sum_{\pi\in S_n}\sgn(\pi)\, T_{\pi(a_1)\cdots\pi(a_n)}.
\ese
Similarly, the \emph{symmetrization} of $T_{a_1,\ldots,a_n}$ is defined as
\bse
T_{(a_1\cdots a_n)}:=\frac{1}{n!}\sum_{\pi\in S_n} T_{\pi(a_1)\cdots\pi(a_n)}.
\ese
\ed
Some special cases are
\bi{C}
T_{[ab]}=\frac{1}{2}(T_{ab}-T_{ba}), \qquad 
T_{(ab)}=\frac{1}{2}(T_{ab}+T_{ba}) \\
T_{[abc]} = \frac{1}{6} (T_{abc}+T_{bca}+T_{cab}-T_{bac}-T_{cba}-T_{acb})\\
T_{(abc)} = \frac{1}{6} (T_{abc}+T_{bca}+T_{cab}+T_{bac}+T_{cba}+T_{acb})
\ei
Of course, we can (anti)symmetrize only some of the indices
\bse
T^{ab}_{\phantom{ab}[cd]e} = \frac{1}{2}(T^{ab}_{\phantom{ab}cde}-T^{ab}_{\phantom{ab}dce}).
\ese
It is easy to check that in a contraction (i.e.\ a sum), we have
\bse
T_{a_1\cdots a_n}S^{a_1\cdots [a_i \cdots a_j] \cdots a_n} = T_{a_1\cdots [a_i \cdots a_j] \cdots a_n}S^{a_1 \cdots a_n} 
\ese
and
\bse
T_{a_1\cdots (a_i \cdots a_j) \cdots a_n}S^{a_1\cdots [a_i \cdots a_j] \cdots a_n} = 0.
\ese

\bq
This can be shown directly using the definition of $\d$. Here, we will instead show it by working in local coordinates.

Recall that, locally on a chart $(U,x)$, we can write any form $\omega\in\Omega^n(M)$ as
\bse
\omega = \omega_{a_1\cdots a_n}\d x^{a_1}\wedge \cdots \wedge \d x^{a_n}.
\ese
Then, we have
\bi{rCl}
\d \omega & = & \d \omega_{a_1\cdots a_n}\wedge\d x^{a_1}\wedge \cdots \wedge \d x^{a_n}\\
 & = & \partial_b \omega_{a_1\cdots a_n}\d x^b\wedge\d x^{a_1}\wedge \cdots \wedge \d x^{a_n},
\ei
and hence 
\bse
\d^2\omega = \partial_c\partial_b\omega_{a_1\cdots a_n}\d x^c\wedge\d x^b\wedge\d x^{a_1}\wedge \cdots \wedge \d x^{a_n}.
\ese
Since $\d x^c\wedge\d x^b = - \, \d x^b\wedge\d x^c$, we have
\bse
\d x^c\wedge\d x^b = \d x^{[c}\wedge\d x^{b]}.
\ese
Moreover, by Schwarz's theorem, we have $\partial_c\partial_b\omega_{a_1\cdots a_n} = \partial_b\partial_c\omega_{a_1\cdots a_n}$ and hence
\bse
\partial_c\partial_b\omega_{a_1\cdots a_n} = \partial_{(c}\partial_{b)}\omega_{a_1\cdots a_n}.
\ese
Thus
\bi{rCl}
\d^2\omega & = & \partial_c\partial_b\omega_{a_1\cdots a_n}\d x^c\wedge\d x^b\wedge\d x^{a_1}\wedge \cdots \wedge \d x^{a_n}\\
& = & \partial_{(c}\partial_{b)}\omega_{a_1\cdots a_n}\d x^{[c}\wedge\d x^{b]}\wedge\d x^{a_1}\wedge \cdots \wedge \d x^{a_n}\\
& = & 0.
\ei
Since this holds for any $\omega$, we have $\d^2=0$. 
\eq

\bc
Every exact form is closed.
\ec

We can extend the action of $\d$ to the zero vector space $0:=\{0\}$ by mapping the zero in $0$ to the zero function in $\Omega^0(M)$. In this way, we obtain the chain of $\R$-linear maps
% \bse
% \begin{tikzcd}
% 0 \ar[r,"\d"]& \Omega^0(M)\ar[r,"\d"]&\cdots\ar[r,"\d"]& \Omega^n(M)\ar[r,"\d"]& \Omega^{n+1}(M)\ar[r,"\d"]& \cdots \ar[r,"\d"]& \Omega^{\dim M}(M)\ar[r,"\d"] & 0
% \end{tikzcd}
% \ese
\bse
0 \xrightarrow{\ \d\ } \Omega^0(M)\xrightarrow{\ \d\ } \Omega^1(M)\xrightarrow{\ \d\ } \cdots \xrightarrow{\ \d\ } \Omega^{n}(M)\xrightarrow{\ \d\ } \Omega^{n+1}(M)\xrightarrow{\ \d\ } \cdots \xrightarrow{\ \d\ } \Omega^{\dim M}(M)\xrightarrow{\ \d\ } 0,
\ese
where we now think of the spaces $\Omega^n(M)$ as $\R$-vector spaces. Recall from linear algebra that, given a linear map $\phi \cl V \to W$, one can define the subspace of $V$
\bse
\ker(\phi) := \{v \in V \mid \phi(v)=0\},
\ese
called the \emph{kernel} of $\phi$, and the subspace of $W$
\bse
\im (\phi) := \{\phi(v)\mid v \in V\},
\ese
called the \emph{image} of $\phi$.

Going back to our chain of maps, the equation $\d^2=0$ is equivalent to
\bse
\im(\d \cl \Omega^n(M)\to\Omega^{n+1}(M)) \se \ker(\d \cl \Omega^{n+1}(M)\to\Omega^{n+2}(M))
\ese
% \bse
% \im(\Omega^n(M)\xrightarrow{\ \d \ }\Omega^{n+1}(M)) \se \ker(\d \cl \Omega^{n+1}(M)\xrightarrow{\ \d \ }\Omega^{n+2}(M))
% \ese
for all $0\leq n \leq \dim M-2$. Moreover, we have
\bi{rCl}
\omega \in \Omega^n(M) \text{ is closed } & \Leftrightarrow\ & \omega \in \ker(\d \cl \Omega^n(M)\to\Omega^{n+1}(M))\\
\omega \in \Omega^n(M) \text{ is exact } & \Leftrightarrow\ & \omega \in \im (\d \cl \Omega^{n-1}(M)\to\Omega^{n}(M)).
\ei
The traditional notation for the spaces on the right hand side above is
\bi{rCl}
Z^n & := & \ker(\d \cl \Omega^n(M)\to\Omega^{n+1}(M)),\\
B^n & := & \im (\d \cl \Omega^{n-1}(M)\to\Omega^{n}(M)),
\ei
so that $Z^n$ is the space of closed $n$-forms and $B^n$ is the space of exact $n$-forms.

Our original question can be restated as: does $Z^n=B^n$ for all $n$? We have already seen that $\d^2=0$ implies that $B^n\se Z^n$ for all $n$ ($B^n$ is, in fact, a vector subspace of $Z^n$). Unfortunately the equality does not hold in general, but we do have the following result.

\bl[Poincar\'e]
Let $M\se \R^d$ be a simply connected domain. Then
\bse
Z^n = B^n \, , \qquad \forall \, n > 0.
\ese
\el

In the cases where $Z^n\neq B^n$, we would like to quantify by how much the closed $n$-forms fail to be exact. The answer is provided by the cohomology group.

\bd
Let $M$ be a smooth manifold. The $n$-th \emph{de Rham cohomology group}\index{cohomology group} on $M$ is the quotient $\R$-vector space
\bse
H^n(M):=Z^n/B^n.
\ese
\ed
You can think of the above quotient as $Z^n/\!\sim$, where $\sim$ is the equivalence relation
\bse
\omega \sim \sigma\ :\Leftrightarrow\ \omega-\sigma \in B^n.
\ese
The answer to our question as it is addressed in cohomology theory is: every exact $n$-form on $M$ is also closed and vice versa if, only if,
\bse
H^n(M) \cong_\mathrm{vec} 0.
\ese
Of course, rather than an actual answer, this is yet another restatement of the question. However, if we are able to determine the spaces $H^n(M)$, then we do get an answer.

A crucial theorem by de Rham states (in more technical terms) that $H^n(M)$ only depends on the global topology of $M$. In other words, the cohomology groups are topological invariants.
This is remarkable because $H^n(M)$ is defined in terms of exterior derivatives, which have everything to do with the local differentiable structure of $M$, and a given topological space can be equipped with several inequivalent differentiable structures. 

\be
Let $M$ be any smooth manifold. We have
\bse
H^0(M) \cong_\mathrm{vec} \R^{\text{(\# of connected components of $M$)}}
\ese
since the closed $0$-forms are just the locally constant smooth functions on $M$. As an immediate consequence, we have
\bse
H^0(\R) \cong_\mathrm{vec} H^0(S^1) \cong_\mathrm{vec} \R.
\ese
\ee

\be
By Poincar\'e lemma, we have
\bse
H^n(M) \cong_\mathrm{vec} 0
\ese
for any simply connected $M\se\R^d$.
\ee




















