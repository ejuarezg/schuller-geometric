\documentclass[a4paper,11pt]{article}
% if your are submitting a pdflatex (i.e. if you have images in pdf, png or
% jpg format)
\pdfoutput=1

% ------------------------------
% Main packages and options
% ------------------------------

% for details on the use of the package, please see the JHEP-author-manual
\usepackage{jheppub}

% if needed
\usepackage[T1]{fontenc}

% ------------------------------
% Extra packages and options
% ------------------------------
% Some packages here are already included by the JHEP package

% Required for alphabetical index at end of document
\usepackage{imakeidx}
\indexsetup{firstpagestyle=empty}
\makeindex[intoc,columns=2,title=Alphabetical Index,options=-s index_style.ist]

% Required for including images
\usepackage{graphicx}
% Set the default folder for images
\graphicspath{{graphics/}}

% For resizing maths relations
\usepackage{relsize}

% Required for manipulating the whitespace between and within lists
\usepackage{enumitem}

% Required for creating figures with multiple parts (subfigures)
\usepackage{subfig}

\usepackage{stmaryrd}

% For including math equations, theorems, symbols, etc.
\usepackage{amsmath,amssymb,amsthm,mathrsfs,amsfonts,xfrac,pifont}

% More descriptive referencing
\usepackage{cleveref}

\usepackage{array,tabularx}
\usepackage[retainorgcmds]{IEEEtrantools}
\usepackage{etoolbox}

\usepackage{xcolor,tikz,pgfplots}
\definecolor{lightergray}{rgb}{0.9,0.9,0.9}

\usepackage{soul}

\usetikzlibrary{matrix,calc,positioning,decorations.markings,decorations.pathmorphing,decorations.pathreplacing}
\usetikzlibrary{arrows,cd}
\usepackage{bbding}
\usetikzlibrary{positioning}
\tikzset{>=stealth}

\usepackage{cancel}
\newcommand\Ccancel[2][black]{\renewcommand\CancelColor{\color{#1}}\cancel{#2}}

\makeatletter
\patchcmd{\@IEEEeqnarray}{\relax}{\relax\intertext@}{}{}
\makeatother

\usepackage{bm}
\usepackage{calc}

\pgfplotsset{compat=1.14}

% ------------------------------
% Theorem styles
% ------------------------------

% Define theorem styles here based on the definition style (used for
% definitions and examples)
\theoremstyle{definition}
\newtheorem*{definition}{Definition}

% Define theorem styles here based on the plain style (used for theorems,
% lemmas, propositions)
\theoremstyle{plain}
\newtheorem{theorem}{Theorem}[section]
\newtheorem{corollary}[theorem]{Corollary}
\newtheorem{lemma}[theorem]{Lemma}
\newtheorem{proposition}[theorem]{Proposition}

% Define theorem styles here based on the remark style (used for remarks and
% notes)
\theoremstyle{remark}
\newtheorem{example}[theorem]{Example}
\newtheorem*{notation}{Notation}
\newtheorem{remark}[theorem]{Remark}
\newtheorem*{solution}{Solution}

% ------------------------------
% Hyperlink style
% ------------------------------

% Override some hyperref preferences
\usepackage{hyperref}
\hypersetup{
  % Uncomment to remove all links (useful for printing in black and white)
  % draft,
  colorlinks=true,
  linkcolor=black,
  breaklinks=true,
  linktoc=all,
  % Already enabled by another package like JHEP package
  % bookmarks=true,
  bookmarksnumbered=true,
  % Link colors
  % urlcolor=webbrown,
  % linkcolor=RoyalBlue,
  % citecolor=webgreen,
  pdftitle={},
  pdfauthor={\textcopyright},
  pdfsubject={},
  pdfkeywords={},
  pdfencoding=unicode,
  pdfstartview={FitH},
  pdfcreator={pdfLaTeX},
}

% ------------------------------
%	Include shortcut definitions
% ------------------------------

\input{shortcuts}

% ------------------------------
% Create title
% ------------------------------

\title{\boldmath Lectures on the Geometric Anatomy\\of Theoretical Physics}

\author{Dr Frederic P. Schuller} 
\affiliation{Friedrich-Alexander-Universit\"at Erlangen-N\"urnberg,\\Institut f\"ur Theoretische Physik III}

% e-mail addresses: one for each author, in the same order as the authors
\emailAdd{fps@aei.mpg.de}

% ------------------------------
% Begin document
% ------------------------------

\begin{document} 

\rule{0cm}{2cm}\\
\includegraphics[width=14cm]{faulogo}
\maketitle

\section*{Introduction}
\addcontentsline{toc}{section}{Introduction}
\markright{Introduction}
\input{ga/00intro}
\newpage

\section{Logic of propositions and predicates}
\input{ga/01logic}
\newpage

\section{Axioms of set theory}
\input{ga/02axioms}
\newpage

\section{Classification of sets}
\input{ga/03classification}
\newpage

\section{Topological spaces: construction and purpose}
\input{ga/04topology}
\newpage

\section{Topological spaces: some heavily used invariants}
\input{ga/05topology_inv}
\newpage
\section{Topological manifolds and bundles}
\input{ga/06manifolds}
\newpage

\section{Differentiable structures: definition and classification}
\input{ga/07differentiable}
\newpage

\section{Tensor space theory I: over a field}
\input{ga/08tensor_i}
\newpage

\section{Differential structures: the pivotal concept of tangent vector spaces}
\input{ga/09tangent_space}
\newpage

\section{Construction of the tangent bundle}
\input{ga/10tangent_bundle}
\newpage

\section{Tensor space theory II: over a ring}
\input{ga/11tensor_ii}
\newpage

\section{Grassmann algebra and de Rham cohomology}
\input{ga/12grassmann}
\newpage

\section{Lie groups and their Lie algebras}
\input{ga/13lie_groups}
\newpage

\section{Classification of Lie algebras and Dynkin diagrams}
\input{ga/14classification}
\newpage

\section{The Lie group \texorpdfstring{$\SL(2,\C)$}{SL(2,C)} and its Lie algebra \texorpdfstring{$\sl(2,\C)$}{sl(2,C)}}
\input{ga/15sl2}
\newpage

\section{Dynkin diagrams from Lie algebras, and vice versa}
\input{ga/16dynkin}
\newpage

\section{Representation theory of Lie groups and Lie algebras}
\input{ga/17representation}
\newpage

\section{Reconstruction of a Lie group from its algebra}
\input{ga/18reconstruction}
\newpage

\section{Principal fibre bundles}
\input{ga/19principal}
\newpage

\section{Associated fibre bundles}
\input{ga/20associated}
\newpage

\section{Connections and connection 1-forms}
\input{ga/21connections}
\newpage

\section{Local representations of a connection on the base manifold: Yang-Mills fields}
\input{ga/22yang-mills}
\newpage

\section{Parallel transport}
\input{ga/23parallel}
\newpage

\section{Curvature and torsion on principal bundles}
\input{ga/24curvature}
\newpage

\section{Covariant derivatives}
\input{ga/25covariant}
\newpage

\setcounter{section}{25}

% \section{Application: Quantum mechanics on curved spaces}
% \input{ga/26quantum}
% \newpage

% \section{Application: Spin structures}
% \input{ga/27spin}
% \newpage

% \section{Application: Kinematical and dynamical symmetries}
% \input{ga/28kinematical}
% \newpage

\section*{Further readings}
\addcontentsline{toc}{section}{Further readings}
\input{biblio}

\printindex

\end{document}
